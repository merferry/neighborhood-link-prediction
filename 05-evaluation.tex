\subsection{Experimental Setup}
\label{sec:setup}

\subsubsection{System used}

We employ a server equipped with two Intel Xeon Gold 6226R processors, each featuring $16$ cores running at a clock speed of $2.90$ GHz. Each core is equipped with a $1$ MB L1 cache, a $16$ MB L2 cache, and a $22$ MB shared L3 cache. The system is configured with $376$ GB RAM and set up with CentOS Stream 8.


\subsubsection{Configuration}

We use 32-bit integers for vertex ids and 32-bit float for edge weights but use 64-bit floats for computations and hashtable values. We utilize $64$ threads to match the number of cores available on the system (unless specified otherwise). For compilation, we use GCC 8.5 and OpenMP 4.5.

%% From LP SRS
We use 32-bit integers for vertex ids and 32-bit float for edge weights but use 64-bit floats for computations and hashtable values. We utilize $64$ threads to match the number of cores available on the system.


\subsubsection{Dataset}

The graphs used in our experiments are given in Table \ref{tab:dataset}. These are sourced from the SuiteSparse Matrix Collection \cite{suite19}. In the graphs, number of vertices vary from $3.07$ to $214$ million, and number of edges vary from $25.4$ million to $3.80$ billion. We ensure edges to be undirected and weighted with a default of $1$.

%% From LP SRS
For the experiments, we use $13$ graphs from the SuiteSparse Matrix Collection\ignore{\cite{suite19}}. In the graphs, number of vertices vary from $3.07$ to $214$ million, and number of edges vary from $25.4$ million to $3.80$ billion. We ensure edges to be undirected and weighted with a default of $1$. For each graph, we generate uniformly random batch of edge deletions of size $10^{-7} |E|$ to $0.1 |E|$. For each batch size, we generate five random batch updates for averaging.

\begin{table}[hbtp]
  \centering
  \caption{List of $13$ graphs obtained from the SuiteSparse Matrix Collection \cite{suite19}, with the directed graphs being marked with an asterisk. Here, $|V|$ is the number of vertices, $|E|$ is the number of edges (after adding reverse edges and removing self-loops), and $D_{avg}$ is the average degree. TOFIX $|E|$.}
  \label{tab:dataset}
  \begin{tabular}{|c||c|c|c|}
    \toprule
    \textbf{Graph} &
    \textbf{\textbf{$|V|$}} &
    \textbf{\textbf{$|E|$}} &
    \textbf{\textbf{$D_{avg}$}} \\
    \midrule
    \multicolumn{4}{|c|}{\textbf{Web Graphs (LAW)}} \\ \hline
    indochina-2004$^*$ & 7.41M & 339M & 45.7 \\ \hline
    uk-2002$^*$ & 18.5M & 561M & 30.3 \\ \hline
    arabic-2005$^*$ & 22.7M & 1.20B & 52.8 \\ \hline
    uk-2005$^*$ & 39.5M & 1.71B & 43.4 \\ \hline
    webbase-2001$^*$ & 118M & 1.86B & 15.8 \\ \hline
    it-2004$^*$ & 41.3M & 2.17B & 52.6 \\ \hline
    sk-2005$^*$ & 50.6M & 3.78B & 74.7 \\ \hline
    \multicolumn{4}{|c|}{\textbf{Social Networks (SNAP)}} \\ \hline
    com-LiveJournal & 4.00M & 69.4M & 17.3 \\ \hline
    com-Orkut & 3.07M & 234M & 76.3 \\ \hline
    \multicolumn{4}{|c|}{\textbf{Road Networks (DIMACS10)}} \\ \hline
    asia\_osm & 12.0M & 25.4M & 2.1 \\ \hline
    europe\_osm & 50.9M & 108M & 2.1 \\ \hline
    \multicolumn{4}{|c|}{\textbf{Protein k-mer Graphs (GenBank)}} \\ \hline
    kmer\_A2a & 171M & 361M & 2.1 \\ \hline
    kmer\_V1r & 214M & 465M & 2.2 \\ \hline
  \bottomrule
  \end{tabular}
\end{table}

% \input{src/fig-leiden-compare}
% \input{src/fig-gve-compare}




\subsection{Comparing Performance of GVE-Leiden}

Upon each graph with edge deletions of batch size $B$ applied, we then predict the same number of edges $B$ using both JC and HP similarity scores. As mentioned in Section \ref{sec:approach}, we avoid high-degree first-order neighbors. For this, we adjust the degree threshold $D$ from $4$ to $32$ in multiples of $2$.

Figure X shows the average runtime of link prediction with degree threshold $D$ of $4$, $8$, $16$, and $32$. Increasing $D$ increases runtime by a small amount, due to increase second-degree neighbor exploration. Increasing/decreasing the batch size however does not have an effect on runtime, as the algorithm must continue to explore possible links with higher scores even if batch/prediction size $B$ edges have been added to the prediction list. Note that runtime for link prediction with JC/HP similarity scores is nearly identical. On a batch size of $0.1 |E|$, we compare the predicted links with the links in the original graphs (without edge deletions). We observe that HP-based link prediction with $D = 16$ obtains the highest average accuracy of $2.4\%$, while HP-based link prediction with $D = 4$ obtains the highest average precision of $6.4\%$. Given the vast set of potential predictions (${}_N C_2 - |E|$), achieving high accuracy and precision remains a challenge.

TODO.

\begin{figure*}[hbtp]
  \centering
  \subfigure[Runtime]{
    \label{fig:input-large--runtime}
    \includegraphics[width=0.98\linewidth]{out/input-large-runtime.pdf}
  }
  \subfigure[Precision]{
    \label{fig:input-large--precision}
    \includegraphics[width=0.98\linewidth]{out/input-large-precision.pdf}
  } \\[0ex]
  \caption{TODO. Input large.}
  \label{fig:input-large}
\end{figure*}

\begin{figure}[hbtp]
  \centering
  \includegraphics[width=0.98\linewidth]{out/strong-scaling-speedup.pdf} \\[-2ex]
  \caption{Overall speedup of our optimized neighbor-based link predictions methods, and its phases (obtaining edges with top-k scores per thread, and merging scores from each thread into a common scoreboard), on batch sizes of $10^{-2}|E|$, with increasing number of threads (in multiples of 2). Increasing the number of threads causes the work in merging phase to increase, thus leading to a poor speedup.\ignore{For this plot, we consider the link-prediction methods \textit{HP4}, \textit{LHN4}, \textit{AA4}, and \textit{RA4}, which perform the best overall. Note that the numerical suffix added to the acronym of each link prediction method indicates the \textit{MAX\_MEDIATOR\_DEGREE} parameter setting.}}
  \label{fig:strong-scaling}
\end{figure}





\subsection{Analyzing Performance of GVE-Leiden}

It is possible that our algorithm only explores one type of connection, but the networks have various types of connections.




\subsection{Strong Scaling of our Optimized Neighbor-based Link Prediction Methods}

Finally, we assess the strong scaling performance of our optimized neighbor-based link prediction methods. In this analysis, we vary the number of thread from $1$ to $32$ in multiples of $2$ for each input graph, and measure the average time taken to predict $10^{-2}|E|$ links by Hub Promoted (HP), Leicht-Holme-Nerman (LHN), Adamic-Adar Coefficient (AA), and Resource Allocation (RA) based link prediction methods with the \textit{MAX\_MEDIATOR\_DEGREE} parameter set to $4$. The results are shown in Figure \ref{fig:strong-scaling}. It not only illustrates the overall scaling performance, but the also scaling of the two phases of each link prediction method, i.e., identifying edges with top-k scores in each thread (scoring phase), and combining the scores obtained by each thread to obtain the global top-k edges (merging phase). With $32$ threads, our optimized link prediction methods achieve an overall speedup of $7.2\times$ (with respect to sequential execution), indicating a performance increase of $1.5\times$ for every doubling of threads. The scalability is limited, as the cost of the merging phase increases with an increase in the number of threads. In fact, at $32$ threads, the merging phase obtains a speedup of $0.7\times$ (yes it is less than $1$\ignore{, as its runtime increases}), while the scoring phase achieves a speedup of $18.5\times$.

\begin{figure*}[hbtp]
  \centering
  \subfigure[Runtime]{
    \label{fig:input-small--runtime}
    \includegraphics[width=0.98\linewidth]{out/input-small-runtime.pdf}
  }
  \subfigure[Precision]{
    \label{fig:input-small--precision}
    \includegraphics[width=0.98\linewidth]{out/input-small-precision.pdf}
  } \\[0ex]
  \caption{TODO. Input small.}
  \label{fig:input-small}
\end{figure*}

