\documentclass[sigconf,nonacm]{acmart}

%% Enable subfigures
\usepackage{subfigure}
%% Enable numbers in scientific format.
\usepackage{siunitx}
%% Enable enumerate start from.
\usepackage{enumitem}

%% Enable theorems
\newtheorem{theorem}{Theorem}[section]
\newtheorem{lemma}[theorem]{Lemma}

%% Enable algorithms
\usepackage{algorithm}
\usepackage[noend]{algpseudocode}
\let\ReturnInline\Return
\renewcommand{\Return}{\State\ReturnInline}
\algrenewcommand\algorithmicrequire{$\rhd$}
\algrenewcommand\algorithmicensure{$\square$}

%% Fonts used in the template cannot be substituted; margin 
%% adjustments are not allowed.
\AtBeginDocument{%
  \providecommand\BibTeX{{%
    \normalfont B\kern-0.5em{\scshape i\kern-0.25em b}\kern-0.8em\TeX}}}

%% Rights management information.
\setcopyright{acmcopyright}
\copyrightyear{2018}
\acmYear{2018}
\acmDOI{XXXXXXX.XXXXXXX}

%% These commands are for a PROCEEDINGS abstract or paper.
\acmConference[Conference acronym 'XX]{Make sure to enter the correct
  conference title from your rights confirmation emai}{June 03--05,
  2018}{Woodstock, NY}
%% Title of the proceedings is different from ``Proceedings of ...''?
% \acmBooktitle{Woodstock '18: ACM Symposium on Neural Gaze Detection,
%  June 03--05, 2018, Woodstock, NY} 
% \acmPrice{15.00}
% \acmISBN{978-1-4503-XXXX-X/18/06}

%% Submission ID.
% \acmSubmissionID{123-A56-BU3}

%% Use the "author year" style of citations and references?
% \citestyle{acmauthoryear}

%% Message
\newcommand{\kk}[1]{{{\color{red} #1}}}
\newcommand{\ds}[1]{{{\color{blue} #1}}}
\newcommand{\su}[1]{{{\color{green} #1}}}

%% Ignore block
\newcommand{\ignore}[1]{}
%% Mark block as done
\newcommand{\ok}[1]{}




\begin{document}

%% Full title of the paper.
\title[A Fast Parallel Approach for Neighborhood-based Link Prediction by Disregarding Large Hubs]{A Fast Parallel Approach for Neighborhood-based \\Link Prediction by Disregarding Large Hubs}

%% Short title to be used in page headers (optional).
% \title[short title]{full title}
% \subtitle{Something other than the title}

%% Authors and their affiliations.
\author{Subhajit Sahu}
\email{subhajit.sahu@research.iiit.ac.in}
\affiliation{%
  \institution{IIIT Hyderabad}
  \streetaddress{Professor CR Rao Rd, Gachibowli}
  \city{Hyderabad}
  \state{Telangana}
  \country{India}
  \postcode{500032}
}

%% Concise author list in page headers.
%\renewcommand{\shortauthors}{Sahu, Kothapalli, and Banerjee, et al.}

%% Show page numbers.
\settopmatter{printfolios=true}

%% Short summary of the work to be presented in the article.
\begin{abstract}
Link prediction can help rectify inaccuracies in various graph algorithms, stemming from unaccounted-for or overlooked links within networks. However, many existing works use a baseline approach, which incurs unnecessary computational costs due to its high time complexity. Further, many studies focus on smaller graphs, which can lead to misleading conclusions. Here, we study the prediction of links using neighborhood-based similarity measures on large graphs. In particular, we improve upon the baseline approach (IBase), and propose a heuristic approach that additionally disregards large hubs (DLH), based on the idea that high-degree nodes contribute little similarity among their neighbors. On a server equipped with dual 16-core Intel Xeon Gold 6226R processors, DLH is on average $1019\times$ faster than IBase, especially on web graphs and social networks, while maintaining similar prediction accuracy. Notably, DLH achieves a link prediction rate of $38.1M$ edges/s and improves performance by\ignore{at a rate of} $1.6\times$ for every doubling of threads.
\end{abstract}

% Link prediction aims to anticipate missing or future connections in a network using known interactions and structure, often employing similarity measures for their simplicity and computational efficiency. While many studies focus on smaller graphs, this report addresses the evaluation of algorithms on larger networks and introduces a heuristic for efficient computation. Additionally, the commonly used baseline approach, despite its simplicity, incurs unnecessary computational costs due to its high time complexity.

%% The code below is generated by the tool at http://dl.acm.org/ccs.cfm.
\begin{CCSXML}
<ccs2012>
<concept>
<concept_id>10003752.10003809.10010170</concept_id>
<concept_desc>Theory of computation~Parallel algorithms</concept_desc>
<concept_significance>500</concept_significance>
</concept>
<concept>
<concept_id>10003752.10003809.10003635</concept_id>
<concept_desc>Theory of computation~Graph algorithms analysis</concept_desc>
<concept_significance>500</concept_significance>
</concept>
</ccs2012>
\end{CCSXML}

% \ccsdesc[500]{Theory of computation~Parallel algorithms}
% \ccsdesc[500]{Theory of computation~Graph algorithms analysis}

%% Pick words that accurately describe the work being presented.
\keywords{Parallel Link prediction, Local/Neighborhood-based}

% \received{20 February 2007}
% \received[revised]{12 March 2009}
% \received[accepted]{5 June 2009}



%% Process the author and title information.
\maketitle

\section{Introduction}
\label{sec:introduction}
Most real-world networks are incomplete \cite{kim2011network, wang2014link}. These networks lie somewhere in the range of a deterministic and a purely random structure, and are thus partially predictable \cite{lu2015toward}. Link prediction is the problem of identifying potentially missing/undiscovered connections in such networks \cite{marchette2008predicting, kim2011network}, or even forecasting future connections \cite{bringmann2010learning, juszczyszyn2011link}, by examining the current network structure\ignore{\cite{zhou2021progresses}}. This is useful in various applications, such as recommending items for online purchase \cite{akcora2011network}, helping people to find potential collaborators \cite{mori2012machine, tang2012cross},\ignore{predicting co-authorships in academic research networks \cite{pavlov2007finding, wohlfarth2008semantic}, identifying abnormal communications \cite{huang2009time}} assessing the trustworthiness of individuals \cite{alnumay2019trust}, uncovering criminal activities and individuals \cite{berlusconi2016link, lim2019hidden}\ignore{, detecting anomalies \cite{huang2006link}}, and predicting new protein-protein interactions or generating hypotheses \cite{cannistraci2013link, nasiri2021novel}.

Similarity measures are frequently employed to predict the likelihood of missing or future links between unconnected nodes in a network \cite{wang2014link, arrar2023comprehensive}. The principle is straightforward: higher similarity indicates a greater likelihood of connection \cite{wang2014link}. The choice of metric depends on the network's characteristics, with no single metric dominating across different datasets \cite{arrar2023comprehensive, zhou2021progresses}. Local / neighborhood-based similarity metrics such as Common Neighbors\ignore{\cite{newman2001clustering}}, Jaccard Coefficient\ignore{\cite{jaccard1901etude}}, S{\o}rensen Index\ignore{\cite{sorensen1948method}}, Salton Cosine similarity\ignore{\cite{salton1973specification}}, Hub Promoted\ignore{ \cite{liben2003link}}, Hub Depressed, Leicht-Holme-Nerman, Adamic-Adar\ignore{\cite{adamic2003friends}}, and Resource Allocation\ignore{\cite{zhou2010solving}}, which are based on neighborhood information within a path distance of two, remain popular \cite{arrar2023comprehensive, wang2014link}. This is due to their simplicity, interpretability \cite{pai2019netdx, barbieri2014follow}, computational efficiency \cite{garcia2014link}, and the ability to capture underlying structural patterns.\ignore{Further, they are often combined with other metrics.}

However, many studies \cite{gatadi2023lpcd, saifi2023fast, benhidour2022approach, mumin2022efficient, rafiee2020cndp, guo2019node, yang2015new, papadimitriou2012fast, wang2019link} and network analysis software \cite{staudt2016networkit, csardi2006igraph} use a baseline approach for link prediction, computing unnecessary similarity scores for all non-connected node pairs. Further, early studies often evaluate a limited number of algorithms on small networks --- this can result in misleading conclusions \cite{zhou2021progresses, zhou2021experimental}. Further, much-existing research does not address link prediction for large networks with close to a billion edges \cite{muscoloni2022adaptive, mumin2022efficient, nasiri2021novel, xian2021towards, ghasemian2020stacking, mara2020benchmarking, wang2019link, xu2019distributed, mohan2017scalable, cui2016bounded, garcia2014link, papadimitriou2012fast, wang2023resisting, wang2023meta, wang2023tdan}. As the collection of data, represented as graphs, reaches unprecedented levels, it becomes necessary to design efficient parallel algorithms for link prediction on such graphs. While link prediction algorithms are often pleasingly parallel, most studies do not address the design of suitable data structures for efficient computation of scores.

Further, the link prediction problem faces significant imbalance, with the number of known links often being several orders of magnitude less than non-existent links. This imbalance hinders the effectiveness of many link prediction methods, particularly on large networks \cite{wang2014link, garcia2014link}. Thus, heuristics are needed to minimize the computation needed, without sacrificing on quality.\ignore{To assess the accuracy of link prediction algorithms, observed links $E$ are split into a training set $E^T$ and a probe set $E^P$ for evaluation.} Quality assessment measures for link prediction include precision, recall, F1 score, accuracy, and Area Under the Receiver Curve (AUC). While AUC is commonly used \cite{arrar2023comprehensive}, it may provide misleading results\ignore{, by giving high scores to algorithms that successfully rank many negatives in the bottom} \cite{yang2015evaluating, lichtnwalter2012link}, leading to our focus on the F1 score.

\ignore{How do you explore the neighbors of each node, and compute intersection? What is a super naive way to do the above?}




\subsection{Our Contributions}

In this\ignore{technical} report, we study parallel algorithms for efficient link prediction in large graphs using neighborhood-based similarity measures. First, we improve upon the baseline approach (IBase). It efficiently finds common neighbors and handles large graphs by tracking top-$k$ edges per-thread and later merges them globally. Next, we propose a novel heuristic approach that additionally discards large hubs (DLH), based on the observation that high-degree nodes contribute poorly to similarity among their neighbors.\footnote{\url{https://github.com/puzzlef/neighborhood-link-prediction-openmp}} We then\ignore{experimentally} determine suitable hub limits, i.e, the degree above which a vertex is considered\ignore{as} a large hub, for link prediction with each similarity metric.

On a machine with two 16-core Intel Xeon Gold 6226R processors, our results show that the DLH approach outperforms IBase by over $1622\times$ and $415\times$ on average with $10^{-2}|E|$ and $0.1|E|$ unobserved edges, respectively. This speedup is achieved while maintaining comparable F1 scores. Notably, DLH achieves a link prediction rate of $38.1M$ edges/s with $0.1|E|$ unobserved edges. We also identified suitable similarity metric for each type of graph. When predicting $0.1|E|$ edges with the DLH approach, we observed that $63\%$ of the runtime is spent on the scoring phase, and especially higher on social networks, with high average degree. With doubling of threads, DLH exhibits an average performance scaling of $1.6\times$.




%% - Use --- for a dash.
%% - Use ``camera-ready'' for quotes.
%% - Use {\itshape very} or \textit{very} for italicized text.
%% - Use \verb|acmart| or {\verb|acmart|} for mono-spaced text.
%% - Use \url{https://capitalizemytitle.com/} for URLs.
%% - Use {\bfseries Do not modify this document.} for important boldface details.
%% - Use \ref{fig:name} for referencing.

%% For a block of pre-formatted text: 
% \begin{verbatim}
%   \renewcommand{\shortauthors}{McCartney, et al.}
% \end{verbatim}

%% For a list of items:
% \begin{itemize}
% \item the ``ACM Reference Format'' text on the first page.
% \item the ``rights management'' text on the first page.
% \item the conference information in the page header(s).
% \end{itemize}

%% For a table:
% \begin{table}
%   \caption{Frequency of Special Characters}
%   \label{tab:freq}
%   \begin{tabular}{ccl}
%     \toprule
%     Non-English or Math&Frequency&Comments\\
%     \midrule
%     \O & 1 in 1,000& For Swedish names\\
%     $\pi$ & 1 in 5& Common in math\\
%     \$ & 4 in 5 & Used in business\\
%     $\Psi^2_1$ & 1 in 40,000& Unexplained usage\\
%   \bottomrule
% \end{tabular}
% \end{table}

%% For a full-width table:
% \begin{table*}
%   \caption{Some Typical Commands}
%   \label{tab:commands}
%   \begin{tabular}{ccl}
%     \toprule
%     Command &A Number & Comments\\
%     \midrule
%     \texttt{{\char'134}author} & 100& Author \\
%     \texttt{{\char'134}table}& 300 & For tables\\
%     \texttt{{\char'134}table*}& 400& For wider tables\\
%     \bottomrule
%   \end{tabular}
% \end{table*}


%% For inline math:
% \begin{math}
%   \lim_{n\rightarrow \infty}x=0
% \end{math},

%% For a numbered equation:
% \begin{equation}
%   \lim_{n\rightarrow \infty}x=0
% \end{equation}

%% For an unnumbered equation:
% \begin{displaymath}
%   \sum_{i=0}^{\infty} x + 1
% \end{displaymath}

%% For a figure:
% \begin{figure}[h]
%   \centering
%   \includegraphics[width=\linewidth]{inc/sample-franklin}
%   \caption{1907 Franklin Model D roadster. Photograph by Harris \&
%     Ewing, Inc. [Public domain], via Wikimedia
%     Commons. (\url{https://goo.gl/VLCRBB}).}
%   \Description{A woman and a girl in white dresses sit in an open car.}
% \end{figure}

%% For a teaser figure.
% \begin{teaserfigure}
%   \includegraphics[width=\textwidth]{sampleteaser}
%   \caption{figure caption}
%   \Description{figure description}
% \end{teaserfigure}


\section{Related work}
\label{sec:related}
Link prediction in network analysis involves diverse algorithms. This include the use of similarity-based methods, dimensionality reduction, and machine learning \cite{arrar2023comprehensive}. As mentioned earlier, the use of similarity measures for link prediction is based on the intuition that the more similar a pair of nodes are, the more likely a link between them, and vice versa. This is consistent with the fact that users tend to create relationships with people who are similar in education, religions, interests and locations \cite{wang2014link}. Similarity measures are commonly classified into three categories: local, quasi-local, and global measures. Local measures calculate scores based on the neighborhood information of nodes with a path distance less than two; global measures use information from the entire network to calculate scores with a path distance greater than two; while quasi-local measures combine local and global measures and calculate scores for nodes with a path distance of no more than two. Other similarity based approaches include the use of random walks and community detection \cite{arrar2023comprehensive}.

Dimensionality reduction techniques \cite{coskun2015link} attempt to map the network's information into a lower dimensional space, while preserving its structural information and components. These include embedding-based methods and matrix factorization based methods. On the other hand, machine learning techniques utilize relevant extracted features from the network data to predict the probability of a link forming between two nodes based on these features. These include the use of supervised \cite{kumari2022supervised, abuoda2020link, lichtenwalter2010new}, unsupervised \cite{rossi2021closing}, reinforcement, and deep learning \cite{cui2018survey, arrar2023comprehensive}.

While the above techniques effectively capture nonlinear relationships between nodes, and enable more accurate predictions within complex networks than similarity based approaches, they come with certain drawbacks. Embedding-based methods face challenges in predicting links for nodes with high centrality --- which have complex connectivity patterns and numerous connections to other nodes \cite{arrar2023comprehensive}. Matrix factorization methods depend heavily on accurately representing the observed network through a low-rank matrix, which may not be feasible for networks with complex structures \cite{martinez2016survey, arrar2023comprehensive}. Further, they require significant computational resources and may result in overfitting if it is not regulated properly, particularly in large-scale networks \cite{arrar2023comprehensive}.

Machine learning based approaches have their own set of drawbacks. Supervised learning based methods require high-quality features to be extracted from the network, which requires domain expertise and can be challenging. As the network evolves, these features may also get outdated. They also need a large labelled dataset, obtaining which can be a time consuming task. Finally, supervised learning techniques have a high time complexity. While deep learning does not require feature extraction, it still requires a substantial about of labelled data, which can be challenging to acquire. Further, the interpretability of deep learning models is limited, and overfitting is also a potential concern.

Similarity-based link prediction methods continue to stay relevant, despite their usually lower prediction accuracy. This is because the size of graphs originated from the web, social networks or biological relations force us to use very simple algorithms if those graphs are to be computed in acceptable time \cite{garcia2014link}. Similarity-based link prediction methods are highly cost-effective, as they offer competitive prediction quality with their low complexity in time and space \cite{zhou2021progresses}. Further, in certain applications like friend recommendation, preference is given to predictions with explanations, a feature not readily achievable through machine learning based techniques \cite{barbieri2014follow}.

%% ON COMMON NEIGHBORS
Yang et al. \cite{yang2015new} introduce the Local Neighbors Link (LNL) measure, motivated by cohesion between common neighbors and predicted nodes, and implemented it on both MapReduce and Spark. Cui et al. \cite{cui2016bounded} present a parallel algorithm for efficiently evaluating Common Neighbors (CN) similarity, obtaining node pairs with CN values surpassing a specified lower bound. Guo et al. \cite{guo2019node} propose Common Neighbour Tightness (CNT), incorporating the aggregation degree of common neighbors by considering their proximity through local information and neighborhood tightness. Rafiee et al. \cite{rafiee2020cndp} introduce Common Neighbors Degree Penalization (CNDP), which factors in clustering coefficient as a structural property and considers neighbors of shared neighbors. Mumin et al. \cite{mumin2022efficient} contribute an algorithm combining common neighbors and node degree distribution to estimate link presence likelihood between two nodes based on local information.

%% ON RANDOM WALK, DIFFUSION
Papadimitriou et al. \cite{papadimitriou2012fast} introduce a similarity-based algorithm employing traversals on paths of limited length, grounded in the small world hypothesis. Their approach extends to directed and signed graphs, with discussions on a potential MapReduce implementation. Kalkan and Hambiralovic \cite{kalkanfinding} propose link prediction based on Personalized PageRank. Vega-Oliveros et al. \cite{vega2021link} investigate the use of susceptible-infected-recovered and independent cascade diffusion models. Their progressive-diffusion (PD) method, founded on nodes' propagation dynamics, provides a stochastic discrete-time rumor model for link prediction.

%% ON COMMUNITY
Mohan et al. \cite{mohan2017scalable} introduce a hybrid similarity measure utilizing parallel label propagation for community detection and a parallel community information-based Adamic-Adar measure, employing the Bulk Synchronous Parallel (BSP) programming model. Wang et al. \cite{wang2019link} propose a link prediction algorithm incorporating an adjustable parameter based on community information (CI), applying it to various similarity indices and a family of CI-based indices. They also develop a parallel algorithm for large-scale complex networks using Spark GraphX. Bastami et al. \cite{bastami2019gravitation} present a gravitation-based link prediction approach, enhancing local and global predictions through the integration of node features, community information, and graph properties. Saifi et al. \cite{saifi2023fast} propose an approach that accelerates link prediction using local and path-based similarity measures by operating on the connected components of a network rather than the entire network.

%% ON APPROXIMATION
Shin et al. \cite{shin2012multi} introduce Multi-Scale Link Prediction (MSLP), employing a tree-structured approximation algorithm for efficient link prediction in large networks. Garcia-Gasulla and Cort{\'e}s \cite{garcia2014link} propose a local link prediction algorithm based on an underlying hierarchical model, emphasizing aspects of parallelization, approximation, and data locality for computational efficiency. Ferreira et al. \cite{ferreira2019scalability} present a multilevel optimization paradigm to enhance the scalability of any link prediction algorithm by reducing the original network to a coarsened version. Benhidour et al. \cite{benhidour2022approach} propose a link prediction method for directed networks, leveraging the similarity-popularity paradigm. The algorithms approximate hidden similarities as shortest path distances, using edge weights that capture and factor out links' asymmetry and nodes' popularity.

%% ON BIPARTITE
\ignore{Aslan and Kaya \cite{aslan2018topic} introduce a similarity-based method using weighted projection to predict potential links between authors and topics in a large-scale bipartite academic information network. Sarhangnia et al. \cite{sarhangnia2022novel} present a similarity measure for link prediction in bipartite social networks, focusing on the neighborhood structure. Shifting to multiplex networks, Sharma and Singh \cite{sharma2016efficient} propose an algorithm for weight prediction using link similarity measures, contributing to the efficient analysis of multiplex network structures. ORCS.}

As mentioned earlier, similarity-based algorithms are competitive to other high-quality dimensionality reduction and machine learning techniques, thanks to their simplicity, interpretability \cite{pai2019netdx, barbieri2014follow}, and computational efficiency \cite{garcia2014link} --- and are thus often combined with other techniques \cite{kumari2022supervised, abuoda2020link, pai2019netdx}. However, the evaluation of these algorithms on large networks is crucial, as testing on small networks can yield misleading conclusions \cite{zhou2021progresses, zhou2021experimental}. Despite this, a significant portion of the discussed works focuses on small \cite{guo2019node, rafiee2020cndp, mumin2022efficient, papadimitriou2012fast, vega2021link, saifi2023fast, ferreira2019scalability, benhidour2022approach} to medium-scale graphs \cite{yang2015new, cui2016bounded, kalkanfinding, mohan2017scalable, wang2019link, bastami2019gravitation, shin2012multi, garcia2014link}, with less than a million or billion edges. Parallelism becomes essential on large networks, and while some approaches, such as the ones based on common neighbors \cite{yang2015new, cui2016bounded}, random walks \cite{papadimitriou2012fast}, community structures \cite{mohan2017scalable, wang2019link}, and approximation \cite{garcia2014link}, incorporate parallelism, the design of suitable data structures for efficient score computation remains an often-overlooked aspect. This technical report aims to bridge both gaps, while also proposing a heuristic for efficient computation.


\section{Preliminaries}
\label{sec:preliminaries}
In an undirected graph $G(V, E)$ with sets of vertices $V$ and edges $E$, link prediction aims to identify missing or future edges from the set $U - E$, where $U$ contains all possible $|V|(|V|-1)/2$ edges in the graph \cite{zhou2021progresses}. Link prediction, thus, is akin to finding a needle in a haystack, as the correct edges need to be identified within a vast set of incorrect ones \cite{garcia2014link, wang2014link}. Garcia et. al \cite{garcia2014link} observe that this ratio goes from $1:11k$ in their best case to $1:27M$ in the worst case. Further, larger graphs more likely to be incomplete. These challenges make the study of link prediction crucial.

Link prediction often relies on similarity metrics between node pairs, reflecting the likelihood of missing or future links \cite{wang2014link, arrar2023comprehensive}. The idea is grounded in the tendency for users to connect with similar individuals. More similarity thus suggests a higher probability of a future link. A ranked list of potential links based on similarity scores can then be used to predict the top-$k$ links are most likely to appear (on were missing) \cite{wang2014link}. Similarity measures are commonly categorized into local, quasi-local, and global measures. Local / neighborhood-based metrics, like Common Neighbours (CN), are calculated based on neighborhood information within a path distance of two. Global indices use network-wide information, while quasi-local indices combine both for distances up to two \cite{arrar2023comprehensive}.

As brought up earlier, despite typically having lower prediction accuracy than machine learning based approaches, similarity-based link prediction methods remain relevant due to the need for simple algorithms in handling large graphs \cite{garcia2014link}. Further, they are highly cost-effective, interpretable, and offer competitive prediction quality with low time and space complexity \cite{zhou2021progresses, barbieri2014follow}.




\subsection{Neighborhood-based Similarity Metrics}

We now discuss nine commonly used local / neighborhood-based similarity metrics for link prediction.


\subsubsection{Common Neighbors (CN)}

This metric, shown in Equation \ref{eq:cn}, counts the shared neighbors between two vertices, $a$ and $b$, in a graph \cite{newman2001clustering}. However, its lack of normalization may pose challenges when comparing node pairs with different degrees of connectivity.

\begin{equation}
\label{eq:cn}
  CN(a, b) = |\Gamma_a \cap \Gamma_b|
\end{equation}


\subsubsection{Jaccard Coefficient (JC)}

The Jaccard Coefficient (JC) \cite{jaccard1901etude} is a popular similarity measure\ignore{in network analysis and link prediction}. It offers a normalized assessment of similarity between nodes based on their neighborhoods. Defined by Equation \ref{eq:jc}, JC assigns higher values to node pairs with a greater proportion of common neighbors relative to their total neighbors.

\begin{equation}
\label{eq:jc}
  JC(a, b) = \frac{|\Gamma_a \cap \Gamma_b|}{|\Gamma_a \cup \Gamma_b|}
\end{equation}


\subsubsection{S{\o}rensen Index (SI)}

S{\o}rensen Index (SI) \cite{sorensen1948method}, also known as S{\o}rensen–Dice coefficient, is another similarity metric commonly applied in network analysis and link prediction. This metric, defined by Equation \ref{eq:si}, extends beyond solely accounting for the size of common neighbors and introduces the idea that nodes with lower degrees are more likely to form links.

\begin{equation}
\label{eq:si}
  SI(a, b) = \frac{|\Gamma_a \cap \Gamma_b|}{|\Gamma_a| + |\Gamma_b|}
\end{equation}


\subsubsection{Salton Cosine similarity (SC)}

The Salton Cosine similarity (SC) \cite{salton1973specification} essentially measures the cosine of the angle between the vectors representing the neighborhoods of nodes $a$ and $b$, as given in Equation \ref{eq:sc}. Similar to other metrics, a higher SC value indicates a greater similarity in the neighborhood structures of the nodes, implying a higher likelihood of a link between them.

\begin{equation}
\label{eq:sc}
  SC(a, b) = \frac{|\Gamma_a \cap \Gamma_b|}{\sqrt{|\Gamma_a| \cdot |\Gamma_b|}}
\end{equation}


\subsubsection{Hub Promoted (HP)}

The HP \cite{liben2003link} metric, defined by Equation \ref{eq:hp}, assesses topological overlap between two nodes in a graph. It is particularly influenced by the lower degree of nodes, and can be valuable in scenarios where the involvement of lower-degree nodes is considered important in understanding network connectivity.

\begin{equation}
\label{eq:hp}
  HP(a, b) = \frac{|\Gamma_a \cap \Gamma_b|}{min(|\Gamma_a|, |\Gamma_b|)}
\end{equation}


\subsubsection{Hub Depressed (HD)}

In contrast to the HP metric, the Hub Depressed (HD) score \cite{zhou2009predicting} is determined by the higher degrees of nodes, as illustrated by Equation \ref{eq:hd}. It can be particularly useful in cases where the influence of highly connected nodes on network structure is of interest.

\begin{equation}
\label{eq:hd}
  HD(a, b) = \frac{|\Gamma_a \cap \Gamma_b|}{max(|\Gamma_a|, |\Gamma_b|)}
\end{equation}


\subsubsection{Leicht-Holme-Nerman (LHN)}

The Leicht-Holme-Nerman (LHN) score \cite{leicht2006vertex} is a similarity metric that assigns high similarity to node pairs that exhibit a greater number of common neighbors than would be expected by random chance. One may use Equation \ref{eq:lhn} to compute the LHN score between two nodes $a$ and $b$ in a graph.

\begin{equation}
\label{eq:lhn}
  LHN(a, b) = \frac{|\Gamma_a \cap \Gamma_b|}{|\Gamma_a| \cdot |\Gamma_b|}
\end{equation}


\subsubsection{Adamic-Adar coefficient (AA)}

AA \cite{adamic2003friends} is a popular measure designed to capture the notion that connections to common neighbors with fewer links are more informative and indicative of similarity between nodes in a network. The formula in Equation \ref{eq:aa} assigns weights inversely proportional to the logarithm of the number of neighbors, reducing sensitivity to highly connected nodes.\ignore{AA highlights that links to less common neighbors provide more discriminative information about node similarity.}

\begin{equation}
\label{eq:aa}
  AA(a, b) = \sum_{c\ \in\ \Gamma_a \cup \Gamma_b} \frac{1}{\log{|\Gamma_c|}}
\end{equation}


\subsubsection{Resource Allocation (RA)}

The RA metric \cite{zhou2010solving} is based on the concept of heat diffusion in a network, emphasizing that heavily connected nodes may not play a critical role in facilitating resource flow between other nodes. Unlike AA, RA penalizes high-degree common neighbors more heavily. The score between nodes $a$ and $b$ is determined by Equation \ref{eq:ra}.

\begin{equation}
\label{eq:ra}
  RA(a, b) = \sum_{c\ \in\ \Gamma_a \cup \Gamma_b} \frac{1}{|\Gamma_c|}
\end{equation}

Note that the CN, AA, and RA metrics lack normalization, and thus only convey ranking information \cite{wang2014link}. In practical applications, one should choose the right metric based on the network's characteristics --- there is no universally dominating metric \cite{zhou2021progresses, ghasemian2020stacking, wang2014link, liben2003link}.




\subsection{Measuring Prediction Quality}

There are a number of metrics used to evaluate the performance of a link prediction model \cite{arrar2023comprehensive}. Performance evaluation metrics can be roughly divided into two categories: threshold-dependent, and threshold-independent metrics \cite{zhou2021progresses}. Precision and recall are the two most widely used metrics in the former category, while Area Under the ROC Curve (AUC), where ROC stands for Receiver operating characteristic, is the most widely used in the later category. To test the algorithm’s accuracy, the observed link, $E$, is divided into two parts: the training set $E^T$ is treated as known information, while the probe set $E^P$ is used for algorithm evaluation, and no information in $E^P$ is allowed to be used for prediction. The majority of known studies applied ‘‘random division’’, namely $E^P$ is randomly drawn from $E$ \cite{zhou2021progresses}.

\ignore{Current link prediction experimental results are usually very low in evaluation metrics values \cite{wang2014link}.}


\subsubsection{Precision}

Precision is a metric used to evaluate the performance of a link prediction model. It measures the proportion of correctly classified positive links (i.e. links that actually exist) among all classified positive links \cite{arrar2023comprehensive}.

Precision is defined as the ratio of relevant items selected to the number of items selected \cite{zhou2021progresses}, i.e., it measures the proportion of correctly classified positive links (i.e. links that actually exist) among all classified positive links \cite{arrar2023comprehensive}. In other words, if we take the top-$L$ links as the predicted ones; among which $L_r$ links are correctly predicted; then, the Precision equals $L_r/L$ \cite{zhou2021progresses}. Equation \ref{eq:precision} gives the formula for computing precision.

\begin{equation}
\label{eq:precision}
  \text{Precision} = \frac{TP}{TP + FP}
\end{equation}


\subsubsection{Recall}

Recall is defined as the ratio of relevant items selected to the total number of relevant items (say $L_r = |E^P|$) \cite{zhou2021progresses}, or the ratio of correctly classified positive links to the total number of positive links \cite{arrar2023comprehensive}. The formula for calculating recall in given in Equation \ref{eq:recall}. A widely adopted way is setting $L = |E^P|$, at which precision = recall \cite{lu2011link, liben2003link, zhou2021progresses}. Although $|E^P|$ is generally unknown, an experiential and reasonable setting is  $|E^P| = 0.1 |E|$ because $10\%$ of links in the probe set are usually enough for us to get statistical solid results while the removal of $10\%$ of links will probably not destroy the structural features of the target network \cite{lu2015toward}.

\begin{equation}
\label{eq:recall}
  \text{Recall} = \frac{TP}{TP + FN}
\end{equation}


\subsubsection{F1 Score}

F1 Score is a measure of the balance between precision and recall \cite{arrar2023comprehensive}. It can calculated as the harmonic mean of percision and recall, as shown in Equation \ref{eq:f1score}.

\begin{equation}
\label{eq:f1score}
  \text{F1 score} = 2 * \frac{\text{Precision} * \text{Recall}}{\text{Precision} + \text{Recall}}
\end{equation}


\subsubsection{Accuracy}

Accuracy is a metric that indicates the percentage of correctly classified links by a link prediction model \cite{arrar2023comprehensive}. It can be calculated using Equation \ref{eq:accuracy}.

\begin{equation}
\label{eq:accuracy}
  \text{Accuracy} = \frac{TP + TN}{TP + TN + FP + FN}
\end{equation}


\subsubsection{Area Under the ROC Curve (AUC)}

Area Under the Receiver (AUC) \cite{hanley1982meaning} is the area under the receiver operating characteristic curve (ROC) is a measure of the prediction algorithm's ability to distinguish between positive and negative links \cite{arrar2023comprehensive}. A higher AUC value indicates a more efficient algorithm compared to random choice \cite{mumin2022efficient}. However, AUC is inadequate to evaluate the early retrieval performance which is critical in real applications \cite{zhou2021progresses}. AUC will give misleadingly overhigh score to algorithms that can successfully rank many negatives in the bottom while this ability is less significant in imbalanced learning \cite{yang2015evaluating, lichtnwalter2012link}.


\ignore{How do you explore the neighbors of each node, and compute intersection? What is a super naive way to do the above?}


\section{Approach}
\label{sec:approach}
\subsection{Optimizations for Local Neighborhood-based Similarity search}
\label{sec:leiden}

Consider an undirected graph $G(V, E)$. For each vertex $u$ in the graph, we count the number of paths to each second order neighbor of $u$, i.e., we calculate $N(u) \cap N(w)$ for each $w \in N(N(u))$. We then use this to calculate two different neighbor similarity metrics, namely, Jaccard's coefficient (JC) and Hub-Promoted (HP) score \cite{gatadi2023lpcd}. By \textit{exploring only second order neighbors} of each vertex, we skip computing scores on pairs of vertices which have no neighbors in common, and thus have a similarity score of $0$. We parallelize this approach with OpenMP's \textit{dynamic} schedule, with a chunk size of $2048$, and optimize path counting and lookup with per-thread collision-free hashtables. Note that we only want to predict links between node pairs with top-$k$ similarity scores, with $k$ being a fraction of the number of edges in the original graph $|E|$. Accordingly, we use a per-thread min-heap based prediction list, which allows us to keep node pairs with top-$k$ scores (per thread), and evict the node pair with the lowest score once we have a node pair with a higher score. Once all vertices have been processed by the threads, we concatenate the prediction lists, sort them by similarity score in increasing order, and return only the top-$k$ links predicted. As an optimization, we convert the per-thread prediction lists into a min-heap only when the prediction lists is populated with $k$ entries. At this stage, predicting links on graph with $2.3$ million edges using $24$ threads takes $14$ seconds.

To further optimize link prediction, we note that low-degree nodes are users who have only a few connections in the social network. These users are more selective in accepting friend requests and are likely to form connections with people they have stronger, more meaningful relationships with, such as close friends and family. Thus, low-degree nodes confer significant similarity among their neighbors, while high-degree nodes generally do not (due to their lack of selectivity). Accordingly, for vertex $u$, we only explore neighbors of $v \in N(u)$ only if $degree(v) \leq D$ (where $D$ is the degree threshold for a neighbor of $u$). With $D = 4$, predicting links on the $2.3$ million edge graph\ignore{ using $24$ threads} now takes only $10$ milliseconds. Figure \ref{fig:about-pruning} shows an explanation of this approach. Here, vertex $4$ is considered for similarity score calculation with vertex $1$, as they are both linked to a common low-degree neighbor, i.e., vertex $2$. However, neighbors of high-degree vertex $3$ are not considered for score calculation.


TODO.

% \input{src/fig-leidenopt-runtime}
% \input{src/fig-leidenopt-modularity}
% \input{src/fig-leiden-pass}




\subsection{Our optimized Leiden implementation}

We now explain the implementation of TODO.


\subsubsection{Main step of GVE-Leiden}

TODO.

% \input{src/alg-leiden}
% \input{src/alg-leidenlm}
% \input{src/alg-leidenre}
% \input{src/alg-leidenag}
\begin{figure*}[hbtp]
  \centering
  \subfigure[Standard approach (IHub)]{
    \label{fig:about-pruning--01}
    \includegraphics[width=0.31\linewidth]{out/about-pruning-01.pdf}
  }
  \subfigure[Disregard hubs with degree $> 8$ (LHub)]{
    \label{fig:about-pruning--02}
    \includegraphics[width=0.31\linewidth]{out/about-pruning-02.pdf}
  }
  \subfigure[Disregard hubs with degree $> 4$ (LHub)]{
    \label{fig:about-pruning--03}
    \includegraphics[width=0.31\linewidth]{out/about-pruning-03.pdf}
  } \\[-2ex]
  \caption{Illustration of our\ignore{neighborhood-based} link prediction approach which disregards large hubs, i.e., first-order neighbors with high degree\ignore{(LHub)}. Here we focus on the neighborhood of vertex $1$\ignore{in the graph}, but the approach applies to each vertex in the graph. In the figure,\ignore{the current vertex} $1$ is outlined in black, its first-order neighbors in red, its second-order neighbors in blue, and explored/traversed vertices are shown with a yellow fill. Edge directions indicate traversal, with some second order vertices omitted for simplicity (dotted edges). (a) Depicts the standard approach (IHub), which considers all second-order neighbors of vertex $1$. (b) Presents our LHub approach, which considers only second-order neighbors linked to $1$ through a small hub (degree $\leq 8$). This pruning reduces runtime and enhances prediction quality. (c) Illustrates LHub approach, where vertices with degree $> 4$ are considered large hubs.}
  \label{fig:about-pruning}
\end{figure*}

\begin{figure*}[hbtp]
  \centering
  \subfigure[Relative runtime (logarithmic scale), of each link prediction method]{
    \label{fig:adjust-mindegree--runtime}
    \includegraphics[width=0.98\linewidth]{out/adjust-mindegree-runtime.pdf}
  }
  \subfigure[F1 score of predicted links (logarithmic scale), of each link prediction method]{
    \label{fig:adjust-mindegree--precision}
    \includegraphics[width=0.98\linewidth]{out/adjust-mindegree-f1score.pdf}
  } \\[0ex]
  \caption{Impact of adjusting the hub limit $L_H$ from $2$ to $1024$ (in multiples of $2$), and to $\infty$, on the runtime (in seconds, log-scale), and precision of predicted links (in percentage, log scale), of each neighbor-based link prediction method, on batch sizes of $10^{-4}|E|$ to $0.1|E|$. The full form of each link prediction method is given in Section \ref{sec:metrics}.}
  \label{fig:adjust-mindegree}
\end{figure*}



\subsubsection{Local-moving phase of GVE-Leiden}

TODO.


\subsubsection{Refinement phase of GVE-Leiden}

TODO.


\subsubsection{Aggregation phase of GVE-Leiden}

TODO.




\subsection{Finding disconnected communities}

TODO.


\section{Evaluation}
\label{sec:evaluation}
\subsection{Experimental Setup}
\label{sec:setup}

\subsubsection{System used}

We employ a server equipped with two Intel Xeon Gold 6226R processors, each featuring $16$ cores running at a clock speed of $2.90$ GHz. Each core is equipped with a $1$ MB L1 cache, a $16$ MB L2 cache, and a $22$ MB shared L3 cache. The system is configured with $376$ GB RAM and set up with CentOS Stream 8.


\subsubsection{Configuration}

We use 32-bit integers for vertex ids and 32-bit float for edge weights but use 64-bit floats for computations and hashtable values. We utilize $64$ threads to match the number of cores available on the system (unless specified otherwise). For compilation, we use GCC 8.5 and OpenMP 4.5.


\subsubsection{Dataset}

The graphs used in our experiments are given in Table \ref{tab:dataset}. These are sourced from the SuiteSparse Matrix Collection \cite{suite19}. In the graphs, number of vertices vary from $3.07$ to $214$ million, and number of edges vary from $25.4$ million to $3.80$ billion. We ensure edges to be undirected and weighted with a default of $1$.

\begin{table}[hbtp]
  \centering
  \caption{List of $13$ graphs obtained from the SuiteSparse Matrix Collection \cite{suite19}, with the directed graphs being marked with an asterisk. Here, $|V|$ is the number of vertices, $|E|$ is the number of edges (after adding reverse edges and removing self-loops), and $D_{avg}$ is the average degree. TOFIX $|E|$.}
  \label{tab:dataset}
  \begin{tabular}{|c||c|c|c|}
    \toprule
    \textbf{Graph} &
    \textbf{\textbf{$|V|$}} &
    \textbf{\textbf{$|E|$}} &
    \textbf{\textbf{$D_{avg}$}} \\
    \midrule
    \multicolumn{4}{|c|}{\textbf{Web Graphs (LAW)}} \\ \hline
    indochina-2004$^*$ & 7.41M & 339M & 45.7 \\ \hline
    uk-2002$^*$ & 18.5M & 561M & 30.3 \\ \hline
    arabic-2005$^*$ & 22.7M & 1.20B & 52.8 \\ \hline
    uk-2005$^*$ & 39.5M & 1.71B & 43.4 \\ \hline
    webbase-2001$^*$ & 118M & 1.86B & 15.8 \\ \hline
    it-2004$^*$ & 41.3M & 2.17B & 52.6 \\ \hline
    sk-2005$^*$ & 50.6M & 3.78B & 74.7 \\ \hline
    \multicolumn{4}{|c|}{\textbf{Social Networks (SNAP)}} \\ \hline
    com-LiveJournal & 4.00M & 69.4M & 17.3 \\ \hline
    com-Orkut & 3.07M & 234M & 76.3 \\ \hline
    \multicolumn{4}{|c|}{\textbf{Road Networks (DIMACS10)}} \\ \hline
    asia\_osm & 12.0M & 25.4M & 2.1 \\ \hline
    europe\_osm & 50.9M & 108M & 2.1 \\ \hline
    \multicolumn{4}{|c|}{\textbf{Protein k-mer Graphs (GenBank)}} \\ \hline
    kmer\_A2a & 171M & 361M & 2.1 \\ \hline
    kmer\_V1r & 214M & 465M & 2.2 \\ \hline
  \bottomrule
  \end{tabular}
\end{table}

% \input{src/fig-leiden-compare}
% \input{src/fig-gve-compare}




\subsection{Comparing Performance of GVE-Leiden}

TODO.

% \input{src/fig-leiden-splits}
% \input{src/fig-leiden-hardness}
\begin{figure}[hbtp]
  \centering
  \includegraphics[width=0.98\linewidth]{out/strong-scaling-speedup.pdf} \\[-2ex]
  \caption{Overall speedup of our optimized neighbor-based link predictions methods, and its phases (obtaining edges with top-k scores per thread, and merging scores from each thread into a common scoreboard), on batch sizes of $10^{-2}|E|$, with increasing number of threads (in multiples of 2). Increasing the number of threads causes the work in merging phase to increase, thus leading to a poor speedup.\ignore{For this plot, we consider the link-prediction methods \textit{HP4}, \textit{LHN4}, \textit{AA4}, and \textit{RA4}, which perform the best overall. Note that the numerical suffix added to the acronym of each link prediction method indicates the \textit{MAX\_MEDIATOR\_DEGREE} parameter setting.}}
  \label{fig:strong-scaling}
\end{figure}





\subsection{Analyzing Performance of GVE-Leiden}

TODO.




\subsection{Strong Scaling of GVE-Leiden}

TODO.


\section{Conclusion}
\label{sec:conclusion}
In an early survey \cite{lu2011link}, we noticed the low stability of individual link predictors and thus suggested ensemble learning as a powerful tool to integrate them. Ensemble learning is a popular method in machine learning, which constructs and integrates a number of individual predictors to achieve better algorithmic performance \cite{zhou2012ensemble}.

While we explore 2nd order neighbors in this report, our techniques can be extended to apply to 3rd or 4th order neighbors.

In the future, we want to explore optimizing quasi-local and global methods of similarity. What are tensor factorization algorithms? Note that local-neighborhood similarity based link prediction techniques can only predict links between vertices that are at a distance of 2 (only). For farther nodes, multiple application of local neighborhood scores may be done (or recursive scoring).


Points:

- Can the algorithms scale to 3rd order neighbors? (is it relevant?)

- What does precision mean in the plots? \% of matching links with ground truth? How many ground truth links were missed?

- Explain the confusion y-axis on precision.

- Explain the algorithm, including the baseline. Tell which parameter setting of max mediator degree is suitable.

- Explain the detailed detailed approach diagram. How do you explore reachable nodes at distance 2. How do you compute intersection (or score for AA, RA)? How do you skip high-degree mediators (or 1st order neighbors)? Why is this efficient?

- How do you maintain per-thread prediction list? Is it efficient? How do you merge the prediction lists? Is it efficient?

- What is the link prediction problem? Why is it similar to a compression/GAN problem? Understand the given network, and predict the missing links. Why it can never be 100\% precise, but precise enough?

- Why the input is only 1 graph? Importance of learning from other similar graphs in the dataset (or from the subsets of the graph). Explainability --- why this is important --- thus the simple algorithms (they help with explainability).

- Sizes of graphs are large so running other approaches is expensive.

- Why precision on road networks and protein k-mer graphs low? (low average degrees)

- What is the processing / prediction rate? (by predicted edges, and by number of edges in the graph)

- Can you show details plots? No.

- Why is string scaling of our link prediction approach low? Why is it especially for the merging phase?

- How do you explore the neighbors of each node, and compute intersection? What is a super naive way to do the above?

- Are there no missing links in the original graphs?

- What is the expected precision of random guess?

- Why not showcase 0.01 E and 0.1 E, and remove the other plots?

- Why the choice of best method appears to change with batch size?

- Arent graphs heterogeneous? Why one method works?

- Why do these methods perform well?

- Why is precision low?

- What we observe on our dataset? What we observe on smaller graphs?

- On a set of smaller graphs, we observe/identify the most precise approaches, and observe the X, X, and X to perform the best. Accordingly, we run these methods on the graphs in our dataset.

- Selecting best approach based on precision and runtime, for batch size $10^{-4}|E|$ to $0.1|E|$, and present the result. The runtime and precision of each method (appendix).

- Comparison with baseline approach ($\infty$), i.e., exploring all possible/reachable pairs at distance 2, and quickly computing intersection, scores, adding to heap, and merging across threads. Explain with figure in approach.

- All approaches require computing intersection, even AA and RA.

- Runtime + speedup + precision for $0.1|E|$.


%% The acknowledgments section.
\begin{acks}
I would like to thank Prof. Kishore Kothapalli, Prof. K. Swarupa Rani, Ashwitha Gatadi, and Balavarun Pedapudi for their support.
\end{acks}

%% Bibliography style to be used, and the bibliography file.
\bibliographystyle{ACM-Reference-Format}
\bibliography{main}

\clearpage
\appendix
% \section{Additional Figures}
\begin{figure*}[hbtp]
  \centering
  \subfigure[Speedup of \textit{LHub} approach with different hub limits $L_H$]{
    \label{fig:adjust-overall--runtime}
    \includegraphics[width=0.48\linewidth]{out/adjust-overall-speedup.pdf}
  }
  \subfigure[F1 score of predicted links (logarithmic scale), with different hub limits $L_H$]{
    \label{fig:adjust-overall--precision}
    \includegraphics[width=0.48\linewidth]{out/adjust-overall-f1score.pdf}
  } \\[0ex]
  \caption{Overall impact of adjusting the hub limit $L_H$ from $2$ to $1024$ (in multiples of $2$), and to $\infty$, on the speedup and F1 score of predicted links (log scale), of neighbor-based link prediction methods, with the number of unobserved edges $E^U$ of $10^{-2}|E|$ and $0.1|E|$. Speedup is measured with respect to hub limit $L_H$ of $\infty$.\ignore{, using geometric mean of runtimes for link prediction using the LHub approach using all similarity scores given in Section \ref{sec:metrics}; while overall F1 score is obtained by taking the average.}}
  \label{fig:adjust-overall}
\end{figure*}


\end{document}
\endinput
%% End of file.




%% NOTES:
%% - Parallelization seems to be not efficient for small batch updates.
%% - Discuss about conflicting updates
%% - 


%% TODO:
%% - Scale up the size of the graphs
%% - Move experiments to a better server
%% - Include a weak- and strong- scalabiilty plot: run the expt from 2 to 128 threads
%% - overall space planning
%% - add a few lines on novelty of the paper.
%% - table comparison of related work
%% - Include a section on preliminaries that talks about the various algorithmic ideas (Louvain, Label Propagation)

%% Workplan:
%% - KK -- Read Introduction, Related Work,
%% - Dip Sankar -- Approach -- summarize the main algorithmic ideas,
%% - Subhajit -- Results -- Plots, scalability, Dataset, experiments, implementation details,
